\documentclass[russian, 12pt]{beamer}
\usepackage{multicol}
\usepackage[T2A,T1]{fontenc}
\usepackage[utf8]{inputenc}
\setcounter{secnumdepth}{3}
\setcounter{tocdepth}{3}
\usepackage{amsmath}
\usepackage{amssymb}
\usepackage{graphicx}
\usepackage{paratype}
\usepackage{hyperref}
\usepackage{xcolor}
\usepackage{listings}

\definecolor{codegreen}{rgb}{0,0.6,0}
\definecolor{codegray}{rgb}{0.5,0.5,0.5}
\definecolor{codepurple}{rgb}{0.58,0,0.82}
\definecolor{backcolour}{rgb}{0.95,0.95,0.92}

\lstdefinestyle{mystyle}{
    backgroundcolor=\color{backcolour},   
    commentstyle=\color{codegreen},
    keywordstyle=\color{magenta},
    numberstyle=\tiny\color{codegray},
    stringstyle=\color{codepurple},
    basicstyle=\ttfamily\footnotesize,
    breakatwhitespace=false,         
    breaklines=true,                 
    captionpos=b,                    
    keepspaces=true,                 
    numbers=left,                    
    numbersep=5pt,                  
    showspaces=false,                
    showstringspaces=false,
    showtabs=false,                  
    tabsize=2
}

\newcommand{\R}{\mathbb{R}}
\newcommand{\A}{\mathcal{A}}
\newcommand{\red}[1]{\textcolor{red!85!black}{{#1}}}
\renewcommand{\sp}[1]{\mathrm{sp}\left\{{{#1}}\right\}}
% Syntax: \colorboxed[<color model>]{<color specification>}{<math formula>}
\newcommand*{\colorboxed}{}
\def\colorboxed#1#{%
  \colorboxedAux{#1}%
} 
\newcommand*{\colorboxedAux}[3]{%
  % #1: optional argument for color model
  % #2: color specification
  % #3: formula
  \begingroup
    \colorlet{cb@saved}{.}%
    \color#1{#2}%
    \boxed{%
      \color{cb@saved}%
      #3%
    }%
  \endgroup
}

\makeatletter
\usepackage{tikz}
\usetikzlibrary{overlay-beamer-styles}
\usepackage{pgfplots}
% Remove warning about running in backwards compatibility mode
\pgfplotsset{compat=1.17}
\usepackage[all,cmtip]{xy}
\usetheme{Pittsburgh}
\usefonttheme{professionalfonts}

\setbeamertemplate{navigation symbols}{%
  \hspace{3.8em}%
  \vspace{0.5em}%
  \usebeamercolor[fg]{structure}%
  \usebeamerfont{subtitle}%
  \insertframenumber%
}
\makeatother
\setbeamercovered{invisible}

\usepackage{babel}

\title{Алгоритмы поиска. Оценка сложности.}
\subtitle{Алгоритмы и структуры данных}
\author{
  Мулюгин Николай
  %\and
  %Кузнецов Максим \texorpdfstring{\thinspace}{Lg}А.
  }
\date{10.09.2022}

\begin{document}

\begin{frame}
\titlepage
\end{frame}

%%%%%%%%%%%%%%%%%%%%%%%%%%%%%%%%%%%%%%%%%%%%%%%%%%%%%%%%%%%%%%%%%%%%%%%%%%%%%%%
\begin{frame}
\frametitle{Мотивация}
  \begin{itemize}
    \item Зачем нам что то искать?\\[0.5cm]
    \pause
    \item Зачем нам что то быстро искать?\\[0.5cm]
    \pause
    \item Где нам искать?\\[0.5cm]
    
    %\item Повседневное применение.\\[0.5cm]

  \end{itemize}
\end{frame}
%%%%%%%%%%%%%%%%%%%%%%%%%%%%%%%%%%%%%%%%%%%%%%%%%%%%%%%%%%%%%%%%%%%%%%%%%%%%%%%
\begin{frame}
\frametitle{Массив}
\begin{center}
  \includegraphics[scale=0.5]{img/array.png}\\  
\end{center}\pause
  \textbf{Массив} - структура данных, хранящая набор значений 
  в памяти непосредственно друг за другом\\[0.3cm]\pause
  \textbf{Индекс массива} - номер элемента в массиве 
  в памяти непосред\\[0.3cm]\pause
  Вопрос: сложно ли получить элемент по индексу?\\ \pause
  Ответ: \;$O(1)$
\end{frame}
%%%%%%%%%%%%%%%%%%%%%%%%%%%%%%%%%%%%%%%%%%%%%%%%%%%%%%%%%%%%%%%%%%%%%%%%%%%%%%%
\lstset{style=mystyle}
\begin{frame}[fragile]
\frametitle{Массив}
\begin{center}
  \includegraphics[scale=0.5]{img/array.png}\\  
\end{center}
Массивы в C++
\begin{lstlisting}[language=C++]
//create array with 4 int type elements
int[4] arr = {5,7,2,3};
//read value at 3
int three = arr[3];//three = 3
//write to 3
arr[3] = 4;//[5 7 2 4]
\end{lstlisting}
\end{frame}
%%%%%%%%%%%%%%%%%%%%%%%%%%%%%%%%%%%%%%%%%%%%%%%%%%%%%%%%%%%%%%%%%%%%%%%%%%%%%%%
\begin{frame}
\frametitle{Домашнее задание}
\begin{itemize}
  \item Написать программу реализующую алгоритм Эратосфена.
  \item Оценить сложность - написать в программе комментариями мысли и результат.
  \item Сделать pull request в папку lecture$\_$01/homework/ с файлом решения.
\end{itemize}
\end{frame}


\end{document}
